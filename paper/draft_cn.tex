\documentclass[UTF8]{ctexart}
\usepackage{graphicx}
\usepackage{booktabs}
\usepackage{hyperref}
\usepackage{geometry}
\geometry{a4paper,margin=1in}

\title{强化适应度驱动的生产线调度优化:多场景下的算法比较与复现}
\author{你的姓名\\你的单位}
\date{\today}

\begin{document}
\maketitle

\begin{abstract}
我们研究了生产线调度中以交付期限为核心约束的优化问题,针对低准时率与高罚金的痛点,提出并系统验证了“强化适应度函数”(提高截止权重与软惩罚系数)的有效性。在 LOOSE/MEDIUM/TIGHT 三类订单紧迫度场景下,我们对 GA、GA+VNS、GA+VNS+SA、PSO 四种算法进行了三场景×30 次实验,比较了利润、产线利用率、准时率与罚金率,并进行了相对 GA 的配对 t 检验。结果表明:强化适应度策略显著提升了准时率并降低罚金;PSO 在期限遵守方面表现稳定与优越,而 GA+VNS+SA 在利用率与利润方面更具均衡性。我们开源了数据与脚本,并给出了参数调整时的最小重跑流程,确保结论可复现。
\end{abstract}

\section{引言}
生产线调度需要平衡交付期限与效率成本。本文的贡献:提出强化适应度函数;构建多场景评价基准;比较 GA 家族与 PSO;提供可复现的实验流程与统计。

\section{方法}
\subsection{指标定义}
利润、利用率、准时率、惩罚率。
\subsection{强化适应度函数}
提高截止权重与软惩罚系数(如 `alpha\_deadline`、`beta\_late\_units`),在搜索早期抑制晚交货,引导解向更高准时率区域;软惩罚的连续刻画利于 GA/VNS/SA 在复杂解空间持续前进。
\subsection{算法配置}
GA、GA+VNS、GA+VNS+SA、PSO;解码策略 EDD;PSO 随机源受 `--seed` 控制以实现复现。

\section{实验设置}
场景为 LOOSE/MEDIUM/TIGHT;每算法每场景 \(n=30\)。统计方法为均值±标准差,显著性采用相对 GA 的配对 t 检验(ns/*/***)。数据与脚本路径:`experiments/batch_*_m6s3_summary.csv`,`experiments/scripts/*.py`。

\section{结果}
\graphicspath{{figures/m8_plots/}}

如图~\ref{fig:m8-profit}、\ref{fig:m8-util}、\ref{fig:m8-ontime}、\ref{fig:m8-radar} 所示,四算法在三场景下的利润、利用率、准时率及综合性能对比如下。

\begin{figure}[htbp]
  \centering
  \includegraphics[width=0.9\linewidth]{m8_profit_comparison.png}
  \caption{利润对比(三场景 × 四算法,n=30)}
  \label{fig:m8-profit}
\end{figure}

\begin{figure}[htbp]
  \centering
  \includegraphics[width=0.9\linewidth]{m8_utilization_comparison.png}
  \caption{利用率对比(三场景 × 四算法,n=30)}
  \label{fig:m8-util}
\end{figure}

\begin{figure}[htbp]
  \centering
  \includegraphics[width=0.9\linewidth]{m8_on_time_rate_comparison.png}
  \caption{准时率对比(三场景 × 四算法,n=30)}
  \label{fig:m8-ontime}
\end{figure}

\begin{figure}[htbp]
  \centering
  \includegraphics[width=0.9\linewidth]{m8_radar_comparison.png}
  \caption{综合性能雷达图}
  \label{fig:m8-radar}
\end{figure}

数值摘要与显著性详见 \href{figures/m8_plots/m8_summary_table.md}{M8 数值摘要表(Markdown)}。

\section{讨论}
强化适应度在不同场景均提升期限遵守;PSO 的标准差趋零反映当前参数与解码稳定最优。若需探索随机性或参数(`w/c1/c2/max\_vel`),建议仅重跑 PSO 三场景×30 次并重新聚合与统计。

\section{结论与未来工作}
当交付及时性为首要目标时推荐 PSO;当利润与资源利用为优先时推荐 GA+VNS+SA。未来将扩展到多目标、动态订单、更多资源约束与实时调度。

\section{复现与开源}
数据与场景:`data/scenarios/*.json`;汇总数据:`experiments/batch_*_m6s3_summary.csv`;脚本:`experiments/scripts/*.py`;最小重跑:仅当 PSO 参数/随机性改变时,重跑三场景×30 次 → 聚合统计 → 同步表格与图。

\bibliographystyle{plain}
\bibliography{references}

\end{document}